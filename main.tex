\documentclass{article}
% \documentclass[]{article}

\usepackage[mitschrift,aufgaben]{kern}
% \usepackage{geometry}
\geometry{
    a4paper, 
    twoside=true, 
    bindingoffset=1cm, 
    left=3.5cm, 
    right=3.5cm, 
    top=3.5cm, 
    bottom=3.5cm
}

\ihead{\textbf{Integrierter Kurs IV}\\\textit{Experimentalphysik II}\\\texttt{Skript}}
\chead{\textit{Tom Folgmann}}

\title{Integrierter Kurs IV}
\author{Experimentalphysik II\\Tom Folgmann}

\begin{document}
    \maketitle
    \tableofcontents

    \section{Atome und Atommodelle}
        \subfile{Notes/lec1.tex}
        \subfile{Notes/lec2.tex}
        \subfile{Notes/lec3.tex}
        \subfile{Notes/lec4.tex}
        \subfile{Notes/lec5.tex}
        \subfile{Notes/lec6.tex}
        \subfile{Notes/lec7.tex}
        \subfile{Notes/lec8.tex}
        \subfile{Notes/lec9.tex}

    % Bohrsches Atommodell, in 9. VL eingeleitet
        \subfile{Notes/lec10.tex}
        \subfile{Notes/lec11.tex}

    % Mitschrift @home
        \subfile{Notes/lec12.tex}
        \subfile{Notes/lec13.tex}

    % Wieder live dabei
        \subfile{Notes/lec14.tex}
        \subfile{Notes/lec15.tex}
        
    % Abschluss Kap 7 und Beginn Kap 8 "Der Spin"
        \subfile{Notes/lec16.tex}
        \subfile{Notes/lec17.tex}
        \subfile{Notes/lec18.tex}

    % Kapitel 9:
    \section{Hyperfeinstruktur und weitere Effekte auf die Energieniveaus des Wasserstoffatoms}
        \subfile{Notes/lec19.tex}

        % Überblick: Spektrale Eigenwertaufspaltung
        \subfile{Ueberblicke/SpektralAufspaltung.tex}

    % Kapitel 10:
    \section{Mehrelektronensysteme}
        \subfile{Notes/lec20.tex}
        \subfile{Notes/lec21.tex}


    \newpage
    \subfile{Anhang/Literatur.tex}
\end{document}