\documentclass{subfiles}

\begin{document}
    \marginnote{\textbf{\textit{VL 20}}\\30.06.2023, 11:45}
    \emph{Heute war der letzte Versuch des IKs, präsentiert von Frau Kiliani.}\\

    Wir können eine Wellenfunktion $\psi\in\mcL^2((\R^3)^2)$ für ein Zweielektronensystem beschreiben durch Seperation: $\psi(x,y) = \varphi_x(x)\cdot\varphi_y(y)$. 

    \begin{Aufgabe}
        \nr{} Verallgemeinere die Aussage auf $n\in\N$ Teilchen. Schreibe das Wahrscheinlichkeitsmaß für zwei Teilchen auf. 
    \end{Aufgabe}
    Da die Teilchen \emph{ununterscheidbar} sein sollen, muss die Wellenfunktion $\psi$ \emph{symmetrisch} sein, also die Wahrscheinlichkeiten sich bei umgedrehter Ortsauswertung gleichen:
    \[
        \abs{\varphi_x(x)\cdot \varphi_y(y)}^2 \stackrel{!}{=} \abs{\varphi_x(y)\cdot \varphi_y(x)}^2.
    \]
    
\end{document}