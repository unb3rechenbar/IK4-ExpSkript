\documentclass{article}

\begin{document}
    \marginnote{\textbf{\textit{VL 7}}\\21.04.2023, 11:45}{-1cm}

    Die induzierte Emission folgt dem Zusammenhang $dN_{I,21} = \lint{B_{21}\cdot u(\nu, t)\cdot N_2}{t}{}$. Im thermischen Gleichgewicht gilt
    \[dN_{12} = dN_{E,21} + dN_{I,21},\]
    also nach Definitionen 
    \[\frac{N_2}{N_1} = \frac{B_{12}\cdot u(\nu,t)}{A_{21} + B_{21}\cdot u(\nu,t)} \Leftrightarrow \frac{N_2}{N_1} = \frac{\exo(E_2/(k_B\cdot T))}{\exp(E_1/(k_B\cdot T))},\]
    bei welchen der Zähler und Nenner sogenannte \href{https://de.wikipedia.org/wiki/Boltzmann-Statistik}{\emph{Boltzmann-Faktoren}} sind. Mit $E_2-E_1 = h\cdot\nu$ folgt dann insgesamt 
    \[u(\nu,t) = \frac{A_{21}}{B_{12}\cdot\exp(h\cdot\nu/(k_B\cdot T)) - B_{21}}.\]
\end{document}