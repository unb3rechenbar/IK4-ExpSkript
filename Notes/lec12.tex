\documentclass{subfile}

\begin{document}
    \subsection{Schrödingergleichung im elektromagnetischen Feld}
    \marginnote{\textit{\textbf{VL 12}\\@home}\\31.05.2023, 08:15}
        Im elektromagnetischen Feld nehmen wir den Hamiltonoperator in der Form
        \[
            H = \fdef{
                \frac{1}{2\cdot m_e}\cdot\Bbra{
                    P(x) + e\cdot A_t(Q)(x)
                }^2 - e\cdot\varphi_t(Q)(x) + V(Q)(x)
            }{x\in\mcH}
        \]
        für ein Vektorpotential $A:\R\to(\R^3\to\R^3)$ und ein Skalarpotential $\varphi:\R^3\to\R$, sowie ein Potantial $V:\R^3\to\R$ an. 

        Der Zusammenhang zu den elektrischen und magnetischen Feldern ist dabei gegeben durch
        \[
            B_t(r) = \rot A_t(r)\qquad E_t(r) = -\dv{t}A_t(r) - d\varphi_t(r)(1).
        \]
        \begin{Aufgabe}
            \nr{} Zeige, daß die felddefinierenden Potentiale $A$ und $\varphi$ eindeutig oder nicht eindeutig definiert sind. 
        \end{Aufgabe}
        Für unser Potential $A$ fordern wir die \emph{Coulomb Eichung} $dA_t(r)(h) = 0$ für beliebige $h\in\R^3$. Für die zeitabhängige Schrödingergleichung $\cmath\hbar\dv{t}\psi(t) = H(\psi(t))$ finden wir nun für ein $\psi:\R\to H^2(\R)$ die Differentialgleichung
        \begin{align*}
            \cmath\cdot\hbar\cdot d{\psi(t)} &= -\frac{\hbar^2}{2\cdot m_e}\cdot d^2(\psi(t)) - \frac{\cmath\cdot e\cdot\hbar}{2\cdot m_e}\cdot A_t\cdot d\psi(t) - \frac{\cmath\cdot e\cdot\hbar}{2\cdot m_e}\cdot d(A_t)\cdot\psi(t) \\
            &\qquad + \frac{e^2}{2\cdot m_e}\cdot \scpr{A_t}{A_t}\cdot\psi(t) - e\cdot\psi(t)\cdot\varphi_t + V\cdot\psi(t)
        \end{align*}
        für ein beliebiges $x\in\R$ in eine Richtung $h\in\R$. Wir wählen hier bei meistens wie in der Physik üblich $h = 1$. 
        \begin{Aufgabe}
            \nr{} Betrachte einmal genau die Dimensionen der einzelnen Summanden und bewerte, inwiefern der Ausdruck sinnvoll bzw. missverständlich ist und ob er sogar korrigiert werden muss.
        \end{Aufgabe}
        \indent Die Coulomb Eichung liefert uns dann die verkürzte Form
        \begin{align*}
            \cmath\cdot\hbar\cdot d{\psi(t)} &= -\frac{\hbar^2}{2\cdot m_e}\cdot d^2(\psi(t)) - \frac{\cmath\cdot e\cdot\hbar}{2\cdot m_e}\cdot A_t\cdot d\psi(t) \\
            &\qquad+ \frac{e^2}{2\cdot m_e}\cdot\scpr{A_t}{A_t}\cdot\psi(t) - e\cdot\varphi_t\cdot\psi(t) + V\cdot\psi(t).
        \end{align*}

    
    \subsection{Semiklassische Wechselwirkung mit dem Strahlungsfeld}
        In der semiklassischen Betrachtung nehmen wir das elektromagnetische Feld als \emph{nicht quantisiert} an, sodaß wir einen Rotator $A$ wählen können. Weiter nehmen wir an, daß $A^2$ vernachlässigbar klein ist und das Feld \emph{quellenfrei} ist, also $\varphi_t(x) = 0$ für alle $t,x\in\R$. Die elektromagnetische Welle ist im physikalischen Anschauungssinne also weit von ihrer Quelle entfernt. Der Störoperator $H_1$ hat demnach dann die Form 
        \[
            H_1 = \fdef{-\frac{\cmath\cdot e\cdot\hbar}{m_e}\cdot A_t(Q)(x)\cdot P(x)}{x\in\mcH}.
        \]
        Stellen wir $A_t$ in Form komplex rotierender Summanden dar, so finden wir
        \[
            A_t(x) = A_0\cdot\exp(\cmath\cdot \bbra{\scpr{k}{x} - \omega\cdot t}) + \conj(A_0)\cdot\exp(-\cmath\cdot\bbra{\scpr{k}{x} - \omega\cdot t}),
        \]
        wobei wir $A_0\in\C^3$ als \emph{Amplitude} und $k\in\R^3$ als \emph{Wellenvektor} bezeichnen. Wir nehmen ferner an $\omega = c_0\cdot\dabs{k}{2}$, die Orthogonalität von $k$ und $A_0$ und die Zerlegung von $A_0$ in $a_0\cdot e_r$ mit $e_r\in\R^3$ als Richtungsvektor der Amplitude, also Polarisationsvektor der Welle. Damit ist der Hamiltonoperator oben gegeben als
        \[
            H(\psi) = -\frac{\cmath\cdot\hbar\cdot e}{m_e}\cdot a_0\cdot\exp(\cmath\cdot\sckr{k}{r})\cdot e_r\cdot d\exp(-\cmath\cdot\omega\cdot t) + \dots,
        \] 
        sodaß der Term $\exp(-\cmath\cdot\omega\cdot t)$ eine \emph{zeitliche Störung} darstellt. 
        \begin{Aufgabe}
            \nr{} Rechne die Punkte \dots in der Gleichung aus. 

            \nr{} Ist die periodische Störung schon in $\Gamma_{a\to b}$ enthalten?
        \end{Aufgabe}
        \noindent Wir stellen eine Proportionalität zwischen $\Gamma_{n\to m}$ und dem Skalarprodukt $\abs{\braopket{\psi_{m,0}}{H_1}{\psi_{n,0}}}$ fest. Es lässt sich zunächst ausformulisieren
        \[
            -\braopket{\psi_{n,0}}{H_1}{\psi_{n,0}} = -\frac{\cmath\cdot\hbar\cdot e}{m_e}\cdot a_0\cdot\scpr{\psi_{n,0}}{\exp(\cmath\cdot\scpr{k}{r})\cdot e_r\cdot d\psi_{n,0}},
        \]
        wobei wir nun den exponentiellen Ausdruck mittels $C^\infty$ Analytizität in einer Potenzreihe zu 
        \[
            \exp(\cmath\cdot\scpr{k}{r}) = 1 + \cmath\cdot\scpr{k}{r} + \frac{(\cmath\cdot\scpr{k}{r})^2}{2} + \dots
        \]
        ausschreiben können. In der sogenannten \emph{Dipolnäherung} vernachlässigen wir alle Terme höherer Ordnung und erhalten die Abschätzung $\exp(\cmath\cdot\scpr{k}{r})\approx 1$. Wenn jedoch die Übergangsrate $\Gamma_{n\to m}$ gegen Null verschwindet, so müssen höhere Terme der Reihe berücksichtigt werden. Hier klassifizieren wir drei Übergangstypen.
        \begin{itemize}[label=$\to$]
            \item \emph{Erlaubte Übergänge (Dipolnäherung)}: Für nicht verschwindendes $\Gamma_{n\to m}$ gilt die Näherung über den ersten Summanden der Potenzreihe.
            \item \emph{Verbotene Übergänge}: Für kleine $\Gamma_{n\to m}$ müssen höhere Summanden der Potenzreihe berücksichtigt werden.
            \item \emph{Streng verbotene Übergänge}: Für $\Gamma_{n\to m} \to 0$ kann keine Näherung mehr vorgenommen werden. 
        \end{itemize}
        Für den ersten Fall der Dipolnäherung können wir weiter schreiben
        \[
            -\braopket{\psi_{n,0}}{H_1}{\psi_{n,0}} \approx -\frac{\cmath\cdot\hbar\cdot e}{m_e}\cdot a_0\cdot\scpr{\psi_{n,0}}{e_r\cdot d\psi_{n,0}} = \frac{e}{m_e}\cdot\scpr{\psi_{n,0}}{e_r\cdot P(\psi_{n,0})}.
        \]
        Mit einem Kommutatortrick können wir den Ausdruck weiter umschreiben und den idealen Hamiltonoperator $H_0$ ins Spiel bringen:
        \begin{align*}
            [Q,P^2] &= P\circ[Q,P] + [Q,P]\circ P = 2\cdot\cmath\cdot\hbar\cdot P, \\
            [Q,H_0] &= \frac{1}{2\cdot m_e}\cdot [Q,P^2] + [Q,V(Q)] = \frac{\cmath\cdot\hbar}{m_e}\cdot P.
        \end{align*}
        Damit können wir den Impulsoperator $P$ in der Form $P = m_e\cdot[Q,H_0]/(\cmath\cdot\hbar)$ ausdrücken. Bringen wir dies ein in die Dipolnäherung, so ergibt sich
        \begin{align*}
            -\braopket{\psi_{n,0}}{H_1}{\psi_{n,0}} &\approx \frac{e}{\hbar}\cdot a_0\cdot\braopket{\psi_{n,0}}{e_r\cdot (Q\circ H_0 - H_0\circ Q)}{\psi_{m,0}} \\ 
            &= \frac{e}{\hbar}\cdot a_0\cdot\braopket{\psi_{n,0}}{e_r\cdot Q\circ H_0}{\psi_{m,0}} - \frac{e}{\hbar}\cdot a_0\cdot\braopket{\psi_{n,0}}{H_0\circ e_r\cdot Q}{\psi_{m,0}}.
        \end{align*}
        Verwenden wir nun die Eigenwerte $E_{n,0}$ zu $\psi_{n,0}$ bzw. $E_{m,0}$ zu $\psi_{m,0}$, so erhalten wir mit $\omega_{n,m}:=E_{n,0} - E_{m,0}$ den Zusammenhang
        \[
            -\braopket{\psi_{n,0}}{H_1}{\psi_{n,0}} \approx -e\cdot a_0\cdot\omega_{n,m}\cdot\braopket{\psi_{n,0}}{e_r\cdot Q}{\psi_{m,0}}.
        \]
        Damit können wir dann schließlich die Proportionalität der Übergangsrate $\Gamma_{n\to m}$ für die Dipolnäherung beschreiben durch
        \[
            \Gamma_{n\to m}^\textit{dipol} \propto e^2\cdot a_0^2\cdot\omega_{n,m}^2\cdot\abs{\braopket{\psi_{n,0}}{e_r\cdot Q}{\psi_{m,0}}}^2.
        \]
        
        \begin{Aufgabe}
            \nr{} Zeige die Konvergenz der behaupteten Reihenentwicklung. 

            \nr{} Überlege dir, wo hier Indizes vertauscht sein könnten. 
        \end{Aufgabe}

    \subsection{Auswahlregeln für Wasserstoff}
        Die Lösungen des Wasserstoffatoms lassen sich geschickterweise durch die Kugelflächenfunktionen $(l,m)\mapsto Y_{l,m}:(\Theta,\varphi)\mapsto\R$ für Winkel $(\Theta,\varphi)\in[0,\pi]\times[0,2\pi]$ und zugehörige Radiusfunktionen $(n,l)\mapsto R_{n,l}:\R\to\R$ darstellen. Für eine Eigenfunktion $\psi_{n,l,m}$ des Wasserstoffatoms $H(\psi) = E\cdot\psi$ findet sich dann der Ausdruck
        \[
            \psi_{n,l,m} = R_{n,l}\cdot Y_{l,m}.
        \]
        Ihre eindeutige Charakterisierung findet über die drei Quantenzahlen $(n,l,m)\in\N\times [n-1]\times D_{m,l}$ mit $D_{m,l}:=\{-l + n:n\in\N_0\}\cap[-l,l]$ statt. Nun wollen wir die Übergänge durch Auswahlregeln klassifizieren.

        Durch Integration über das Argument $(r,\Theta,\varphi)\in\R_{>0}\times(0,\pi),(-\pi,\pi)$ können wir für einen Übergang $a:=(n_a,l_a,m_{l,a})\to (n_b,l_b,m_{l,b})=:b$ als Slalarprodukt unter Verwendung der Seperationsmöglichkeit für entsprechende Zustände $\psi_a,\psi_b$ schreiben
        \begin{align*}
            M_{n\to m} := \braopket{\psi_a}{H}{\psi_b} &= \int_{\R_{>0}}R_a\cdot R_b\abs{\det(Jf)} \cdot \int_{(0,\pi)}\tilde P_a\cdot\tilde P_b\cdot \abs{\det(Jf)} \\
            &\qquad\cdot \int_{(-\pi,\pi)}\exp(-\cmath\cdot(m_a - m_b))\cdot\abs{\det(Jf)}.
        \end{align*}
        Hierbei ist zunächst besonders das letzte Integral interessant, denn es verschwindet überhaupt der gesamte Ausdruck nicht, wenn das letzte Integral nicht verschwindet und $m_a = m_b$ gilt. Es gibt hier noch eine zweite Möglichkeit, auf welche wir später zu sprechen kommen; hier folgern wir zuerst noch die erste \emph{Auswahlregel} für Wasserstoff:
        \begin{center}
            $\Delta m = m_b - m_a = 0 \implies $linear polarisiertes Licht. 
        \end{center}
        \begin{Aufgabe}
            \nr{} Berechne das Matrixelement $M_{n\to m}$. Nutze hierzu die Aufspaltung $Y_{l,m} = \tilde P_{l,m}(\cos(\Theta))\cdot\exp(\cmath\cdot m\cdot\varphi)$ - die \emph{normierten Legendre Polynome}. Zerlege dann das Skalarprodukt in die entsprechenden Integrale und zeige die Nullhypothese. 
        \end{Aufgabe}
        Um den Zusammenhang für \emph{zirkulär polarisiertes Licht} deutlicher hervorzuheben, wollen wir auf die genauen Integralfaktoren optisch verzichten und uns die Proportionalitäten genauer ansehen. Wir finden durch die Definition der zirkulär polarisierten Welle durch zwei linearkombinierte lineare Wellen mit Phasenverschiebung $\pi/2$ zunächst die Form 

        \[
            M_{a\to b}^\pm \propto \int_{(-\pi,\pi)}\exp(\cmath\cdot(m_a - m_b)\cdot\varphi)\cdot \ubra{(\cos(\varphi)\pm\cmath\cdot\sin(\varphi))}{\exp(\pm\cmath\cdot\varphi)}\;d\varphi.
        \]
        Daraus können wir durch das Integral $\int \exp(\cmath\cdot (m_a - m_b \pm 1)\cdot x)\;dx$ die Folgerung der zweiten und dritten Auswahlregel vollziehen:
        \begin{center}
            $\Delta m = \pm 1 \implies$ zirkulär polarisiertes Licht.
        \end{center}
        Aus physikalischer Sicht bezwecken die Auswahlregeln die Einhaltung der \emph{Drehimpulserhaltung} und der \emph{Paritätserhaltung}. Im Falle der zirkulären Polarisation entlang der $e_3$ Achse muss sich bei einem abgesonderten Photonendrehimpuls $\pm\hbar$, also $\sigma^\pm$ Licht, der Atomdrehimpuls um die Negation $\mp\hbar$ ändern. Für linearpolarisiertes Licht können wir aus diesem Konzept die Änderung des Atomdrehimpulses um $0$ begründen, da die Überlagerung von $\sigma^+$ mit $\sigma^-$ die linear polarisierte Welle $\pi$ ergibt. 

        Bei unserer Überlegung über das Verschwinden des Matrixelementes $M_{n\to m}$ sprachen wir bereits eine zweite Möglichkeit an: Der Raumwinkel $\Theta$. Wir überlegen uns ein Szenario: Für ein Integral der Form $\int_\R f$ gilt die Gleichheit zu $0$, wenn $f$ zum Beispiel die Bedingung der \emph{Ungeradheit} erfüllt. Liegt diese Charakteristik vor, so können wir das Integral sauber aufspalten zu $\int_{\R_{<0}}f + \int_{\R_{\geq 0}}f$ und erhalten durch das Auslöschen der Terme das Ergebnis $0$. Das spezifische Verhalten der Funktion $f$ bei Spiegelung des Arguments $x$ zu $-x$ bezeichnen wir als \emph{Parität}. Sie ist in positiv und negativ unterteilt mit den Definitionen 
        \[
            f(x) = +f(-x)\qquad f(x) = -f(-x).
        \]
        Da wir das Verschwinden des Integrales oben gerade nicht erzielen wollen, fordern wir für den Integranden seine \emph{Geradheit}. Wir können dies verschärfen auf die Zustände $\psi_a,\psi_b$: Wir fordern von ihren Darstellungen in $\tilde P_a$ und $\tilde P_b$ \emph{unterschiedliche Parität}. Damit erhalten wir mit der Darstellung $\tilde P_{l,m}(\pi - \Theta,\varphi + \pi) = (-1)^l\cdot\tilde P_{l,m}(\Theta,\varphi)$ die Forderung \emph{bei einem Übergang muss $l$ ungerade werden} die vierte, $l$ betreffende Auswahlregel:
        \begin{center}
            $\Delta l = \pm 1$, denn der Drehimpuls des Photons ist $\pm\hbar$.
        \end{center}
        Zuletzt halten wir noch fest, daß es für die Hauptquantenzahl $n$ keine solchen Auswahlregeln gibt. Für die Spinquantenzahl $s$ werden wir noch $\Delta s = 0$ feststellen, für $j$ finden wir noch $\Delta j \in\{-1,0,1\}$, wobei der direkte Übergang $j = 0\to 0 = j$ \emp{verboten} ist.
        \begin{Aufgabe}
            \nr{} \emph{Weil bei $\Delta s = 0$ die Änderung von $\Delta l = \pm 1$ durch $\Delta m_s \mp 1$ kompensiert werden kann.} Ordne diese Aussage im Kontext der letzten Auswahlregeln ein. 

            \nr{} Bestimme alle Möglichkeiten, wie der Matrixterm $M_{a\to b}$ verschwinden kann, indem du seine Faktoren untersuchst. 
        \end{Aufgabe}


\end{document}