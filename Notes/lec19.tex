\documentclass{subfiles}

\begin{document}
    \marginnote{\textit{\textbf{VL 19}}\\29.06.2023, 10:00\\(l8)}

    \subsection{Hyperfeinstruktur}  


    \subsection{Das vollständige Kernschema}
        Wir schulden dem Modell nun noch wenige Korrekturen, wobei wir die Spin-Bahn-Kopplung bereits diskutiert haben:
        \begin{enumerate}[label=(\roman*)]
            \item Die relativistische Korrektur des Elektrons im Coulomb Potential.
            \item Die Verschmierung der Elektronenladung $-e$ über ein Volumen $\lambda_C^3 = \bbra{\hbar/(m_e\cdot c_0)}^3$.
        \end{enumerate}
        Zum ersten Punkt finden wir über die Energie $E = \sqrt{p^2c_0^2 + m^2c_0^4} - m_c^2 + E_\textit{pot}$ die Beziehung
        \[
            E = \frac{p^2}{2m} + E_\textit{pot} - \frac{p^4}{8m^3c_0^2} + \ldots = E_\textit{nrel} - \Delta E_\textit{rel}.
        \]
        \begin{Aufgabe}
            \nr{} Leite die Energiebeziehung für das relativistische Elektron im Coulomb Potential mittels Reihenentwicklung her. 
        \end{Aufgabe}
        Dabei können wir $\Delta E_\textit{rel} = \langle n,l,m|H'|n,l,m\rangle$ bestimmen, indem wir den Hamiltonoperator $H' :=- \frac{P^4}{8m^3c_0^2}$ definieren. 
        
        \[
            V\mapsto\int_V E_\textit{kin}(r)\mu(dr)
        \]
        \begin{Aufgabe}
            \nr{} Entwickele die Funktion $E_\textit{kin}$ um $r\in\R^3$ und klassifiziere die ersten drei Terme durch (i) ungestört, (ii) Kugelsymmetrie und (iii) Korrektur. 
        \end{Aufgabe}

\end{document}