\documentclass{subfiles}

\begin{document}
    \marginnote{\textbf{\textit{VL 18}}\\28.06.2023}

    \subsubsection*{Betrachtung im schwachen Magnetfeld}
        In einem Magnetfeld, in welchem die Spin Bahn Kopplung noch erhalten ist, gilt für ein Magnetfeld in $z$-Richtung $B = (0,0,B_z)$ der Drehimpuls $J_3 = m_j\cdot\hbar$ mit $m_j\in[j]_\Z$. Der zeitliche Mittelwert von $\mu_j$ ist damit $\langle\mu_j\rangle_3 = -m_j\cdot g_j\cdot\mu_B$, wobei $\mu_B$ das \emph{Bohrsche Magneton} und $g_j$ der \emph{Landé-Faktor} ist. 
        
    \subsubsection*{Spektroskopische Notation}
        Das Termschema ist eine Darstellungsmethode der Energieniveaus eines Atoms inklusive (aller) erlaubten Energieübergänge. Diese Energieniveaus werden dabei kompakt in Tupelform $(S_n,\mcJ_n,L)\in\R^3$ geschrieben, wobei $S$ den \emph{Gesamtspin}, $\mcJ$ den \emph{Gesamtdrehimpuls} und $L$ den \emph{Bahndrehimpuls} angibt. Dabei kann analog auch $(S_n,\mcJ_n,\mcB_L)$ als Zahl- Buchstabentupel notiert werden, wobei $\mcB$ die Zuordnungen $0\mapsto S$, $1\mapsto P$, $2\mapsto D$ usw. bezeichnet. Man notiert dann als \emph{Termsymbol} einen Zustand der Form
        \[
            (S_n,J_n,L) \mapsto ^{2\cdot S + 1}(\mcB_L)_J. 
        \]
        Für Wasserstoff ergibt sich beispielsweise $S = 1/2$, $L = 0$ und $J = 1/2$ zu $^2S_{1/2}$. Für $L = 1$ und $J = 3/2$ ergibt sich $^4P_{3/2}$.

    \subsection{Der Paschen-Back-Effekt}
        Wir wechseln nun in das Setting eines starken Magnetfeldes. Hierbei ist die Spin-Bahn-Kopplung nicht mehr erhalten, da das Magnetfeld die Drehimpulsoperatoren dominiert. Damit orientieren sich $L$ und $S$ unabhängig voneinander im Magnetfeld, $j$ fällt damit als Beschreibungsparameter weg. Die spektrale Energieaufspaltung ist dann gegeben durch
        \[
            \Delta E = \bbra{g_l\cdot m_l + g_s\cdot m_s}\cdot\mu_B\cdot B.
        \]
        

    
\end{document}